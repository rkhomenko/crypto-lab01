\documentclass[a4paper,12pt]{article}
\usepackage[T2A]{fontenc}
\usepackage[utf8]{inputenc}
\usepackage[english,russian]{babel}
\usepackage{amsmath}
\usepackage{geometry}

\geometry{left=1cm}
\geometry{right=1cm}
\geometry{top=1cm}
\geometry{bottom=1.5cm}

\begin{document}

\begin{titlepage}
\begin{center}
    {Московский авиационный институт} \\
    {(национальный исследовательский университет)} \\
    {Кафедра 806}

\vspace{8cm}
\large{
    {Лабораторная работа №1} \\
    {по курсу <<Криптография>>} \\
    {Тема: <<Методы факторизации натуральных чисел>>}
}
\end{center}

\vspace{6cm}
\begin{flushright}
\begin{minipage}{0.4\textwidth}
    \begin{flushleft}
        {Выполнил: студент группы 8О-308} \\
        {Хоменко Роман Дмитриевич} \\
        \vspace{0.5cm}
        {Перподаватель:} \\
        {аспирант кафедры 806} \\
        {Борисов Август Валерьевич} \\
        \vspace{0.5cm}
        Оценка:
    \end{flushleft}
\end{minipage}
\end{flushright}

\vfill
\begin{center}
    {Москва, 2018}
\end{center}

\end{titlepage}

\section{Постановка задачи}

\subsection{Задача}
Факторизовать два натуральных числа, в соответствии с вариантом
задания. Для решения задачи предпочтительнее использовать
субэкспонециальные методы факторизации. Разрешается использовать
сторонние библиотеки и утилиты.

\subsection{Вариант 20}
$n_1 = 284994967805859272853477327862245466978346919806585432133556769959269315271111$ \\
$n_2$ =
        191624208718068015686171299450972805253515909112884480565867902529671655940443 \\
        466481172561918665272590132577464901759414478836063740717847693631691522075814 \\
        453568196437131165707175097041470721811222228045395187521359163973501984457964 \\
        262201487421259483804145780046492118234512749646088825008417181554035121174581 \\
        354219296962410856750448190529031735941575253507798593150790972216736431298009 \\
        9834023023021212767107040301344392783417575981002593796696074442689507301

\section{Метод факторизации}

\subsection{Выбор метода}

В соответствии с условием задачи, решено было использовать субэкспоненциальный метод
факторизации. В начале выбор пал на алгоритм Ленстры с эллиптическими кривыми.
Готовых реализаций нашлось не много. Все они не были достаточно быстрыми.
На ЭВМ с AthlonII они факторизовывали первое число, в среднем, за 16 часов.
Такое время работы алгоритма говорит о его неоптимальной реализации.

Было решено подобрать другой субэскспоненциальный алгоритм с
оптимальной реализацией в виде библиотеки или утилиты. Данным
критериям полностью удовлетворял общий метод решета числового
поля (GNFS). Нашлось несколько реализаций утилит для различных операционных
систем. Выбор пал на CADO-NFS. Данная утилита имеет очень широкие возможности для
настройки использования ресурсов системы, а также множество готовых конфигураций
для факторизации чисел различной длины. Еще одно достоинство данной утилиты --
возможность распределенных вычислений с помощью системы серверов. К сожалению,
данная возможность не была использована, потому что первое число достаточно
быстро факторизуется, а для второго она не поможет, так как времени, отведенного
на выполнение лабораторной работы, было слишком мало относительно времени факторизации
второго числа даже с помощью нескольких серверов.

\subsection{Общий метод решета числового поля}

Пусть $n$ -- нечетное целое число, $P_d(x)$ -- неприводимый полином степени $d \geq 3$, и
$\theta$ -- целое алгебраическое число, являющееся полиномом степени $d$.

\begin{enumerate}
    \item Выберем степень неприводимого многочлена $d \geq 3$.

    \item Выберем целое число $m$,
          $\left \lfloor n^{1/(d+1)} \right \rfloor < m < \left \lfloor n^{1/d} \right \rfloor$, и $n$ разложим по основанию $m$:
        \begin{equation} \label{eq:eq_1}
          n = m^d + a_{d-1}m^{d-1} + \ldots + a_0.
        \end{equation}

    \item С разложением \ref{eq:eq_1} связем неприводимый в кольце $Z[x]$
    (кольцо полиномов от переменной $x$ с целыми коэффициентами) полином
    \begin{equation} \label{eq:eq_2}
        f_1(x) = x^d + a_{d-1}x^{d-1} + \ldots + a_0.
    \end{equation}

    \item Определим полином просеивания $F_1(a, b)$ как однородный полином
    от двух переменных $a$ и $b$:
    \begin{equation} \label{eq:eq_3}
        F_1(a, b) = b^d f_1(a / b) = a^d + a_{d - 1}a^{d -1}b
        + a_{d - 2} a^{d-2}b^2 + \ldots + a_0b.
    \end{equation}
    Значения $F_1(a, b)$ равно норме полинома $a - bx$ в алгебраическом числовом
    поле $Q[\theta]$.

    \item Также определим второй многочлен $f_2(x) = x - m$ и соответствующий
    однородный многочлен $F_2(a, b) = a - bm$. Главное требование к выбору
    пары $(f_1, f_2)$ состоит в том, что должно выполнятся условие:
    \begin{equation} \label{eq:eq_4}
        f_1(m) \equiv f_2(m) \mod n
    \end{equation}

    \item Выберем два положительных числа $L_1$ и $L_2$, которые определяют некоторую
    прямоугольную область $SR = \left \{1 \leq b \leq L_1, -L_2 \leq a \leq L_2 \right \}$,
    называемую областью просеивания (sieve region).

    \item Пусть $\theta$ -- корень многочлена $f_1(x)$. Рассмотрим кольцо
    многочленов $Z[\theta]$ (практически корень $\theta$ не вычисляется, а используется только
    для формального описания алгоритма). Определим \textit{алгебраическую факторную базу $FB_1$},
    состоящую из многочленов первого порядка вида $a - b\theta$ с нормой \ref{eq:eq_3},
    являющейся простым числом. Такие многочлены являются простыми неразложимыми элементами
    в кольце алгебраических чисел поля $K = Q[\theta]$. Абсолютные величины норм многочленов
    из факторной базы $FB_1$ ограничим сверху некоторой константой $B1$.

    \item Одновременно определим \textit{рациональную факторную базу} $FB_2$,
    состоящую из всех простых чисел, ограниченных некоторой константой $B_2$.

    \item Произвольный элемент $a$ кольца $K$ называется квадратичным вычетом,
    если существует такой элемент $x \in K$, что $x^2 = a$. Чтобы иметь возможность
    проверять на заключительной стадии алгоритма, является ли найденный в ходе просеивания
    многочлен полным квадратом, определим сравнительно небольшое множество многочленов
    1 порядка $c - d\theta$, норма которых также является простым числом.
    Оно должно удовлетворять условию $FB_1 \cup FB_3 = \O$ и называется \textit{факторной
    базой квадратичных характеров (the Quadratic Character Base)}.

    \item Далее выполняется одновременное просеивание многочленов
    $ \left \{a - b\theta | (a, b) \in SR \right \}$ по факторной
    базе $FB_1$ и целых чисел
    $ \left \{a - bm | (a, b) \in SR \right \}$ по факторной базе $FB_2$
    с целью получения множества $M$, состоящего из \textit{гладких пар (a, b)}.
    Пара $(a, b)$ называется гладкой, если НОД$(a, b) = 1$, и полином $a - b\theta$
    и число $a - bm$ раскладываются полностью по соответствующим факторным базам
    $FB_1$ и $FB_2$. Число гладких пар должно превышать суммарную мощность
    трех факторных баз как минимум на две единицы.

    \item На следующем шаге ищется подмножество $S \subseteq M$ такое что
    $$\prod_{(a, b) \in S}Nr(a - b\theta) = H^2, H \in Z,$$
    $$\prod_{(a, b) \in S}(a - bm) = B^2, B \in Z.$$
    Для нахождения множества $S$ составляется система линейных алгебраических
    уравнений с коэффициентами из множества $F_2 = \left \{ 0, 1 \right \}$,
    решением которой и будут номера множества $S$.

    \item Далее формируем многочлен
    \begin{equation} \label{eq:eq_5}
        g(\theta) = (f_1'(\theta) \prod_{(a, b) \in S} (a - b\theta)),
    \end{equation}
    где $f_1'(x)$ -- производная многочлена $f_1(x)$.

    \item Если вся процедура была выполнена корректно, то многочлен
    $g(\theta)$ является полным квадратом в кольце полиномов $Z[\theta]$.
    Извлекаем квадратные корни из многочлена $g(\theta)$ и целого числа $B^2$,
    находя некоторый многочлен $\alpha(\theta)$ и число $B$.

    \item Заменяем многочлен $\alpha(\theta)$ числом $\alpha(m)$. Отображение
    $\phi : \theta \rightarrow m$ является кольцевым гомеоморфизмом кольца
    алгебраических целых чисел $Z_k$ в кольцо $Z$, откуда получим соотношение:
    \begin{equation} \label{eq:eq_6}
        A^2 = g(m)^2 \equiv \phi(g(\alpha)^2) \equiv
        \phi\left((f_1'(\theta))^2 \prod_{(a, b) \in S}(a - b\theta)\right) \equiv
    \end{equation}
    \begin{equation}
        \equiv (f_1'(m))^2 \prod_{(a, b) \in S}(a - bm) \equiv (f_1'(m))^2 C^2 \mod n
    \end{equation}

    Определив $B = f'(m) \cdot C$, найдем пару целых чисел $(A, B)$,
    удовлетворяющих условию $A^2 \equiv B^2 \mod n$.

    Теперь можно найти делитель числа $n$, вычисляя НОД$(n, A \pm B)$.
\end{enumerate}

Приведенное выше краткое описание алгоритма опускает многие сложные
аспекты его реализации: выбор и построение алгебраической факторной базы,
построение алгебраической базы характеров, расчет необходимого числа гладких
пар, просеивание в решете числового поля, формирование и решение системы линейных
уравнений, вычисление квадратного корня. Однако, оно позволяет полностью описать
метод решета числового поля, поместив его в отчет по лабораторной работе.
Более подробное описание алгоритма можно найти в книге Ш.Т.Ишмухаметова
<<Методы факторизации натуральных чисел>>.

\subsection{Метод факторизации второго числа}
Факторизация натуральных чисел очень тесно связана
с алгоритмом шифрования RSA. Именно отсутствие менее
чем субэкспоненциальных методов факторизации обеспечивает
его надежность. Под эгидой RSA было запущено соревнование
по факторизации чисел. Максимальное факторизованое число
имело 768 бит. Легко видеть, что второе число данной лабораторной работы
гораздо больше, что намекает полную несостоятельность применения
даже метода решета числового поля для его факторизации.

Самой первой, однако не самой логичной, приходит мысль о том,
что числа $n_1$ и $n_2$ связаны каким-либо образом. Однако, найти эту связь
мне не удалось, наиболее вероятно, по причине её отсутствия.

Следующей попыткой стала факторизация второго числа методом
Ферма, так как в результате факторизации первого числа
были получены $p_1$ и $q_1$, которые были достаточно близки.
В предположении того, что это будет также и для второго числа,
пробуем метод Ферма. И ничего. Он отрабатывался очень долго даже для
первого числа.


Последней попыткой стала проверка атаки на RSA при малых
значениях секретной экспоненты. По теореме Винера, должно
выполняться неравенство $$q \le p \le 2q.$$

Если из данного неравенства извлечь корень и домножить на $\sqrt(p)$,
то получим
$$\sqrt{n} \le p \le \sqrt{2n},$$
что позволяет перебрать $p$. Тоже мимо.

\section{Результаты}
\subsection{Первое число}
Первое число было разложено на два множителя
$p = 397695326178862814397952263440193307813$ и
$q = 716616336792661370154476211778412420347$.

Метод эллиптических кривых (с неоптимальной реализацией) справился
за 54632 секунды, а метод решета числового поля за 292.

\subsection{Второе число}
Факторизовать второе число не удалось.

\section{Выводы}
Подводя итог, следует еще раз вспомнить сложность проблемы факторизации
натуральных чисел. Число RSA-768, на данный момент, является предельным.
Факторизовать большие числа за адекватное время до сих пор не удалось.
Таким образом, сложность методов факторизации является основой
для алгоритма RSA.

Однако, сложность методов факторизации будет снижатся. Уже
сейчас появился гораздо более эффективный алгоритм для
квантовых компьютеров. Но можно расслабиться. Для больших
чисел данный алгоритм требует большого числа связанных кубитов.
Гораздо больше, чем доступно на данный момент.

\end{document}
